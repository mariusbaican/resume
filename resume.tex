\documentclass[10pt, letterpaper]{article}

% Packages:
\usepackage[
    ignoreheadfoot, % set margins without considering header and footer
    top=2 cm, % seperation between body and page edge from the top
    bottom=2 cm, % seperation between body and page edge from the bottom
    left=2 cm, % seperation between body and page edge from the left
    right=2 cm, % seperation between body and page edge from the right
    footskip=1.0 cm, % seperation between body and footer
    % showframe % for debugging 
]{geometry} % for adjusting page geometry
\usepackage{titlesec} % for customizing section titles
\usepackage{tabularx} % for making tables with fixed width columns
\usepackage{array} % tabularx requires this
\usepackage[dvipsnames]{xcolor} % for coloring text
\definecolor{primaryColor}{RGB}{0, 0, 0} % define primary color
\usepackage{enumitem} % for customizing lists
\usepackage{fontawesome5} % for using icons
\usepackage{amsmath} % for math
\usepackage[
    pdftitle={Marius Baican's CV},
    pdfauthor={Marius Baican},
    pdfcreator={LaTeX with RenderCV},
    colorlinks=true,
    urlcolor=primaryColor
]{hyperref} % for links, metadata and bookmarks
\usepackage[pscoord]{eso-pic} % for floating text on the page
\usepackage{calc} % for calculating lengths
\usepackage{bookmark} % for bookmarks
\usepackage{lastpage} % for getting the total number of pages
\usepackage{changepage} % for one column entries (adjustwidth environment)
\usepackage{paracol} % for two and three column entries
\usepackage{ifthen} % for conditional statements
\usepackage{needspace} % for avoiding page brake right after the section title
\usepackage{iftex} % check if engine is pdflatex, xetex or luatex

% Ensure that generate pdf is machine readable/ATS parsable:
\ifPDFTeX
    \input{glyphtounicode}
    \pdfgentounicode=1
    \usepackage[T1]{fontenc}
    \usepackage[utf8]{inputenc}
    \usepackage{lmodern}
\fi

\usepackage{charter}

% Some settings:
\raggedright
\AtBeginEnvironment{adjustwidth}{\partopsep0pt} % remove space before adjustwidth environment
\pagestyle{empty} % no header or footer
\setcounter{secnumdepth}{0} % no section numbering
\setlength{\parindent}{0pt} % no indentation
\setlength{\topskip}{0pt} % no top skip
\setlength{\columnsep}{0.15cm} % set column seperation
\pagenumbering{gobble} % no page numbering

\titleformat{\section}{\needspace{4\baselineskip}\bfseries\large}{}{0pt}{}[\vspace{1pt}\titlerule]

\titlespacing{\section}{
    % left space:
    -1pt
}{
    % top space:
    0.3 cm
}{
    % bottom space:
    0.2 cm
} % section title spacing

\renewcommand\labelitemi{$\vcenter{\hbox{\small$\bullet$}}$} % custom bullet points
\newenvironment{highlights}{
    \begin{itemize}[
        topsep=0.10 cm,
        parsep=0.10 cm,
        partopsep=0pt,
        itemsep=0pt,
        leftmargin=0 cm + 10pt
    ]
}{
    \end{itemize}
} % new environment for highlights


\newenvironment{highlightsforbulletentries}{
    \begin{itemize}[
        topsep=0.10 cm,
        parsep=0.10 cm,
        partopsep=0pt,
        itemsep=0pt,
        leftmargin=10pt
    ]
}{
    \end{itemize}
} % new environment for highlights for bullet entries

\newenvironment{onecolentry}{
    \begin{adjustwidth}{
        0 cm + 0.00001 cm
    }{
        0 cm + 0.00001 cm
    }
}{
    \end{adjustwidth}
} % new environment for one column entries

\newenvironment{twocolentry}[2][]{
    \onecolentry
    \def\secondColumn{#2}
    \setcolumnwidth{\fill, 4.5 cm}
    \begin{paracol}{2}
}{
    \switchcolumn \raggedleft \secondColumn
    \end{paracol}
    \endonecolentry
} % new environment for two column entries

\newenvironment{threecolentry}[3][]{
    \onecolentry
    \def\thirdColumn{#3}
    \setcolumnwidth{, \fill, 4.5 cm}
    \begin{paracol}{3}
    {\raggedright #2} \switchcolumn
}{
    \switchcolumn \raggedleft \thirdColumn
    \end{paracol}
    \endonecolentry
} % new environment for three column entries

\newenvironment{header}{
    \setlength{\topsep}{0pt}\par\kern\topsep\centering\linespread{1.5}
}{
    \par\kern\topsep
} % new environment for the header

\newcommand{\placelastupdatedtext}{% \placetextbox{<horizontal pos>}{<vertical pos>}{<stuff>}
  \AddToShipoutPictureFG*{% Add <stuff> to current page foreground
    \put(
        \LenToUnit{\paperwidth-2 cm-0 cm+0.05cm},
        \LenToUnit{\paperheight-1.0 cm}
    ){\vtop{{\null}\makebox[0pt][c]{
        \small\color{gray}\textit{Last updated in September 2024}\hspace{\widthof{Last updated in September 2024}}
    }}}%
  }%
}%

% save the original href command in a new command:
\let\hrefWithoutArrow\href

% new command for external links:


\begin{document}
    \newcommand{\AND}{\unskip
        \cleaders\copy\ANDbox\hskip\wd\ANDbox
        \ignorespaces
    }
    \newsavebox\ANDbox
    \sbox\ANDbox{$|$}

    \begin{header}
        \fontsize{25 pt}{25 pt}\selectfont Marius Baican

        \vspace{5 pt}

        \normalsize
        \mbox{Bucharest}%
        \kern 5.0 pt%
        \AND%
        \kern 5.0 pt%
        \mbox{\hrefWithoutArrow{mailto:marius.baican18@gmail.com}{marius.baican18@gmail.com}}%
        \kern 5.0 pt%
        \AND%
        \kern 5.0 pt%
        \mbox{\hrefWithoutArrow{tel:+40755-934-835}{+40755934835}}%
        \kern 5.0 pt%
        \AND%
        \kern 5.0 pt%
        % \mbox{\hrefWithoutArrow{https://yourwebsite.com/}{yourwebsite.com}}%
        % \kern 5.0 pt%
        % \AND%
        \kern 5.0 pt%
        \mbox{\hrefWithoutArrow{https://linkedin.com/in/marius-baican/}{linkedin.com/in/marius-baican/}}%
        \kern 5.0 pt%
        \AND%
        \kern 5.0 pt%
        \mbox{\hrefWithoutArrow{https://github.com/mariusbaican}{github.com/mariusbaican}}%
    \end{header}

    \vspace{5 pt - 0.3 cm}


    \section{Introduction}

        \begin{onecolentry}
            Computer Science undergraduate student with a keen eye for detail and a strong interest in building reliable, scalable software. Driven by clean design, thoughtful testing, and systems that work and keep working.
        \end{onecolentry}

    \section{Education}
        
        \begin{twocolentry}{
            Sept 2023 – Present
        }
            \textbf{University Politehnica of Bucharest}, Bachelor's Degree in Computer Science\end{twocolentry}


    \section{Experience}
        
        \begin{twocolentry}{
    June 2023 – Present
}
    \textbf{Robotics Team Mentor}, BrickBot -- Focșani, VN
\end{twocolentry}

\vspace{0.10 cm}
\begin{onecolentry}
    \begin{highlights}
        \item Introduced principled design practices for both hardware and software systems
        \item Created custom learning resources to onboard and support new team members
        \item Strengthened leadership and mentoring skills through active team guidance and process coordination
    \end{highlights}
\end{onecolentry}


    
    \section{Projects}

        \begin{twocolentry}{
    \href{https://brickbot.vercel.app/en/home}{brickbot.vercel.app}
}
    \textbf{BrickBot Robotics Team Website}
\end{twocolentry}

\vspace{0.10 cm}
\begin{onecolentry}
    Designed and developed the official website for the robotics team I mentor, aimed at attracting collaborators, increasing visibility, and presenting our projects to the public.

    \begin{highlights}
        \item Implemented a modular architecture with reusable React components and server-side rendering via Next.js
        \item Focused on clean and responsive UI/UX aligned with team brandings
        \item Tech Stack: Tailwind, TypeScript, React, Next.js, HTML, CSS, Vercel
    \end{highlights}
\end{onecolentry}

\vspace{0.2 cm}

 \begin{twocolentry}{
    \href{https://brickbot.vercel.app/docs}{brickbot.vercel.app/docs}
}
    \textbf{BrickBot Documentation Website}
\end{twocolentry}

\vspace{0.10 cm}
\begin{onecolentry}
    Designed and developed the official website for the robotics team I mentor, aimed at attracting collaborators, increasing visibility, and presenting our projects to the public.
    \begin{highlights}
        \item Built using MkDocs and Markdown for a lightweight, fast-loading structure
        \item Customized theme and navigation for clarity and responsiveness
        \item Integrated GitHub Pages for seamless CI/CD deployment
        \item Structured project data with YAML for scalable content management
        \item Tech Stack: Markdown, CSS, MkDocs, GitHub Pages
    \end{highlights}
\end{onecolentry}

\vspace{0.2 cm}

\begin{twocolentry}{
    \href{https://github.com/SimplicityFTC/SimplicityFTC-Quickstart}{github.com/SimplicityFTC/}
}
    \textbf{SimplicityFTC Open-Source Programming Library}
\end{twocolentry}

\vspace{0.10 cm}
\begin{onecolentry}
    Built a performance-oriented robotics library for teams in the FIRST Tech Challenge, simplifying robot programming while integrating powerful, competition-ready tools.
    \begin{highlights}
        \item Implemented a structured logging system for in-match and test diagnostics
        \item Added read/write caching to reduce hardware latency and CPU load
        \item Developed a command-based framework for modular, readable control logic
        \item Created a Bézier curve-based autonomous path follower with smooth motion
        \item Integrated Motion Profiling and a PDFS (Proportional, Derivative, Feedforward and Static terms) controller
        \item Tech Stack: Java
    \end{highlights}
\end{onecolentry}



    
    \section{Technologies}

        \begin{onecolentry}
            \textbf{Languages:} C, Java, HTML, CSS, JavaScript, TypeScript
        \end{onecolentry}

        \vspace{0.2 cm}

        \begin{onecolentry}
            \textbf{Technologies:} Tailwind CSS, React.js, Next.js, Vercel, Linux, Git, GitHub Pages
        \end{onecolentry}


    

\end{document}